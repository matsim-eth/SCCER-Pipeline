\section{Introduction}

It is increasingly recognized that both the environmental and social costs of travel need to be internalized to meet the demand on already strained transport networks by encouraging shifts in travel patterns. 
In this direction, there is a growing body of evidence that informal feedback on energy use can encourage more efficient behavior, both regarding home energy use \citep{faruqui2010impact} and travel behavior \citep{taniguchi2003psychological, fujii2006determinants}.
However, providing feedback on external costs in transport is particularly challenging, due to the heterogeneous nature of the users and privacy constraints.

%%Start here on externalities - Joe, already written
The main externalities of transportation can be divided into two groups: Those that affect other road users, namely congestion, and those that affect non-road users such as noise and emissions \cite{button2004rationale}.
The impact of congestion is primarily the loss of time spent waiting in traffic, where as emissions and noise have both environmental and health consequences.
When a person chooses to drive on a road, they only have to pay for the private time and car usage costs.
However, they do not have to pay the \textit{marginal social cost} (MSC) of their trip.
That is, the additional costs the driver imposes on other drivers by increasing demand on the route.
From economic theory, internalising this MSC in the form of a tax would reduce congestion and provide an overall social benefit \cite{arnott1994economics, pigou2013economics}). 

Micrsosimulation provides advantages from both the supply (network representation) and demand (individual agents) sides when modeling road pricing.
\citet{arnott2001economic} notes that traditional macroscopic models focus on link congestion, while ignoring or simplifying other elements of congestion such as nodal congestion, parking and interactions with pedestrians and spillback.
Microsimulated models allow for the representation of non-homogenous driver behaviour and preferences.
In particular, the importance of value of time heterogenity among individuals in road pricing models has been recognised by numerous researchers \citet{small2001value, verhoef2004product}.
Modern traffic microsimulation frameworks such as MATSim \cite{balmer2009matsim} are to capture range of externalities.

\citet{kaddoura2015marginal} developed an agent based marginal cost-pricing approaches for congestion, emissions and noise exteralities, and applied numerous models successfully to a large scale scenario of Greater Berlin.
When considering the internalisation of congestion costs, a particular contribution of his work, was to assign the external congestion costs to the causing agents.
In particular, \citet{kaddoura2015marginal} notes that it is simple to calculate the incurred congestion for each agent, but much more challenging to map it back to the causing agent.
The approach calculates each agent's contribution to the delays on traveled links using a queue based node-link model including spillback.

In real networks such an approach requires knowing the location and VTTS of every driver connected to a particular incident of congestion.
This is clearly unrealistic. Instead, this work presents an approach to impute externalities using just the GPS trace of the trip, a representation of the road network and aggregated congestion values generated from an agent-based MATSim simulation for Switzerland.

\subsection{Swiss MATSim Scenario}
%%Chris or Joe
The IVT 2015 Baseline Scenario \cite{boesch2015ivt} represents a typical working day in Switzerland for the year 2015.
As a MATSim scenario, the population consists of individual agents, each with daily travel plans (preferences), and social-demographic characteristics.
These agents represent the entire population of Switzerland on the network from \cite{boesch2015network}.
Of particular importance to this work, the 2015 scenario extends the work of \cite{Balmer2007switzerland} by including households and their incomes. 

\subsection{Swiss norms, ARE report, HBEFA}
%%Chris
The Swiss Federal Office of Spatial Development (ARE) published in 2016 an updated external costs and benefits analysis of transport in Switzerland \cite{are2016externalcosts}.
This analysis presents the most recent external costs and benefits calculations for the entire Swiss transport system, primarily focusing on external environmental, health and accident-related costs, utilizing the same methodology as \cite{ecoplaninfras2014externeeffekte}.
Specifically, external costs for 12 different cost categories are computed, differentiated by transport mode, type and user. 

\cite{foen2010pollutants} provides an updated detailed analysis of past and predicted future pollutant emissions due to road transport in Switzerland from 1990 to 2035.
The emission factors used in calculating these emissions are based on Version 3.1 of the Handbook for Emission Factor Analysis.
The HBEFA database contains emission factors for different vehicle categories and traffic situations, differentiated by emission type, pollutant and year.

An additional supplementary report on external traffic delay costs between 2010 and 2014 was published by \cite{mkinfras2016staukosten}.
The objectives of this study were to estimate vehicle hours of delay and which proportion of these delays can be attributed to heavy vehicles.
For 2013, this is achieved by first combining and aligning both traffic flow data obtained from INRIX for 2013 and traffic demand data from the National Passenger Transport Model.
The time lost per road section is then calculated by comparing the actual travel times to free-flow travel times, where traffic jams are considered to occur only when the actual speed is less than 65\% of the free-flow speed.
For the others year, online data from the Federal Roads Office (FEDRO) counting stations was used.
Their estimated for the years 2009 to 2014 are reported in \Cref{tab:vehHoursDelayMkInfras}.
Since these values were computed using a so-called "at-least" approach, they will tend to underestimate the actual lost time and associated delay costs.
%This is particularly the case for non-motorways road segments, where long road lengths, flawed speed data and lack of zero mean speed cases play a significant role.

\createtable%
{Updated estimated vehicle hours of delay for 2009-2014}%
{Updated estimated vehicle hours of delay for 2009-2014 and comparison to previously reported values from ARE 2012}%
{\label{tab:vehHoursDelayMkInfras}}%
{%
  \begin{tabular}[c]{lrrrrrrrr}
    \toprule
    \multirow{3}{*}{} & & \multicolumn{7}{c}{Vehicle hours of delay (Mio/a)}\\ 
    & & \multicolumn{2}{c}{Motoway} & \multicolumn{2}{c}{Non-motoway} & \multicolumn{3}{c}{Total}\\
    & &  LMV & HMV & LMV & HMV & LMV & HMV & Both\\
    \midrule
    \multirow{6}{*}{Updated}
    & 2009 & 11.41 & 0.53 & 11.23 & 0.22 & 22.64 & 0.75 & 23.39 \\
	& 2010 & 15.19 & 0.79 & 11.23 & 0.22 & 26.42 & 1.00 & 27.42 \\
	& 2011 & 15.68 & 0.81 & 11.23 & 0.22 & 26.90 & 1.03 & 27.93 \\
	& 2012 & 16.45 & 0.85 & 11.23 & 0.22 & 27.68 & 1.07 & 28.75 \\
	& 2013 & 15.67 & 0.82 & 11.23 & 0.22 & 26.90 & 1.04 & 27.93 \\
	& 2014 & 16.62 & 0.87 & 11.23 & 0.22 & 27.85 & 1.09 & 28.94 \\
	\midrule
	Previous & 2009 & 12.40 & 1.08 & 14.10 & 0.28 & 26.5 & 1.4 & 27.9 \\
    \bottomrule
  \end{tabular}
}%
{\cite{mkinfras2016staukosten}, p.141, Annexe A4}


\red{TODO : Something about norms?}


\subsection{Green Class Data}
%%Joe
In the Green Class 2016 pilot project, the SBB created a new mobility offering for subscribers which consisted of a General Abonnenment (GA), an electric BMW i3, park+ride subscription, car and bike sharing subscriptions.
The 139 participants were tracked for the duration of the project using a mobile app enabled with GPS.

\red{how much detail on the work from geoINFK?} These GPS traces were processed to identify individual trips and modes of travel \cite{rabaul2016greenclassprocessing}.

The trip records contain the following information:
\begin{itemize}
  \item user id
  \item trip id
  \item start timestamp
  \item end timestamp
  \item identified mode of travel
\end{itemize}

Each GPS waypoint records contains: 
\begin{itemize}
  \item longitude
  \item latitude
  \item timestamp
  \item accuracy (meters)
  \item user id
\end{itemize}

\red {summary of trips, waypoints in table}

In total, for the 139 participants, there were _______ trips recorded, with an an average of ______ per person per day (see Table \ref{Table:green_class}). 
Figure \ref{figure:waypoint_dist} presents the distribution of the number of waypoints per trip. 
A large number of GPS points is important for accurate route detection and identification of the link entry and exit timings.

A core component of the pilot project was the availability of an electric car to subscribers, powered by renewable energy.
Asumming similar driving behaviour independent of the drive-train type, the emissions avoided for each trip can calculated.
For this analysis, the dataset was filtered to include only trips with a start and finish within switzerland using either the electric car or personal automobile. 

\paragraph{GPS Waypoints}

