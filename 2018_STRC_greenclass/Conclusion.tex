\section{Conclusion}
%%Joe
This paper presented a methodology for imputing the externalities on GPS traces using the MATSim framework. 
The 2015 MATSim Switzerland scenario was used to provide hourly aggregate estimates for incurred and caused congestion, and pollutant emission factors were taken from the Handbook for Emission Factor Analysis (HBEFA). 
The suitability of the MATSim scenario for this purpose was evaulated by validating the Switzerland wide externaties against published offical values. 
The agent-based aspect of MATSim allows for much finer calculation of externalities by taking into account the heterogenity in both the population and travel behaviour. 
The validation indicated that the 2015 scenario is mostly suitable for such purposes with some caveats. 
Firstly, the emissions results are highly dependent on the make-up of the national car fleet. 
As such further work will incorportate a car ownership model for Switzerland into the scenario.
Secondly, the total delay hours in the scenario are lower than the offical numbers for motorways, but higher for other roads. 
While the errors are most likely introducted from simplifications on both sides, more work needs to be done to identify the sources of these discrepancies.
The analysis of the SBB Green Class project with the proposed methodology illustrate the environmental benefits of electric vehicles in mobility schemes, even when convential alternatives are still available. Ongoing work will explore the experienced and caused congestion of subscribers, and identify how such information can be communicated to respondents to influence thier travel behaviour. 

\section{Aknowledgement}
This research was supported by the Swiss National Research Foundation and the Swiss Compentency Center for Energy Research.
